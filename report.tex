% Please do not change the document class
\documentclass{scrartcl}

% Please do not change these packages
\usepackage[hidelinks]{hyperref}
\usepackage[none]{hyphenat}
\usepackage{setspace}
\doublespace %turn on after writing!

% You may add additional packages here
\usepackage{amsmath}

% Please include a clear, concise, and descriptive title
\title{Psalms 120:2}

% Please do not change the subtitle
\subtitle{COMP240 - CPD Report}

% Please put your student number in the author field
\author{1604281}

\begin{document}

\maketitle

\section{Introduction}

This report will take a (possibly troublingly deep) dive into the challenges that have presented themselves in my own work and abilities, and the steps I’ll be taking to overcome them and continue to improve. With the end of the year fast approaching, time runs; sand through the glass of memory. With an indeterminable void of potential careers lurking in the near future, it is important that I work on improving my professional and personal skills as much as possible. This report will look at key skills across five domains (Interpersonal, Dispositional, Cognitive, Procedural and Affective) and investigate the challenges I have faced in relation to these skills.


\section{Giving Critique}

Assessing and reviewing group project work is an important aspect of the Plan-Do-Test-Review cycle where concerns can be raised over any work that could be improved. Although I feel fairly comfortable giving criticism or advice on improvements in people’s work, I tend to consider it less effort to leave it be and allow other members of the team to raise concerns. This is obviously not ideal, as it can lead to the issue being raised much later, or not raised at all. I believe I largely tend to avoid giving criticism because the involved discussions it can lead to often distract me from what I am doing, and can also be draining.

\subsection{Action}

To remedy this, at each team playtest of the game, I will give at least one piece of honest criticism if I feel there are any issues. I will use this as an opportunity to see if the rest of the team consider this a problem as well.


\section{Projects I Enjoy Less}

While I am fairly good at managing my time, one exception to this is situations where I enjoy some projects far less than others. In these situations, although I still spend the same amount of time doing work overall, I tend to spend far more time working on the projects I find enjoyable, while often putting off and ignoring the ones I find tedious or boring. Because of this, the quality of my projects sometimes varies significantly due to the lack of time put into the ones that disinterest me. This is obviously a problem not just in university, but in the software industry in general, as you may often be given projects that you have little personal interest in that must be completed on time.


\subsection{Action}

As an achievable way to reduce this problem, I will divide my studio practice hours each week equally between my projects. Although I may still spend more time on certain projects while working at home, this will at least ensure that I spend a decent amount of time on all of my work.


\section{Unreal Actor System}

For our Third Year game project, our team is leaning towards using Unreal, instead of Unity as we did this year. Compared to Unity, my knowledge of Unreal is severely limited and I am far less comfortable with the engine. Even if the team eventually decide to continue with Unity, Unreal is Industry-Standard software that would be highly beneficial to be comfortable with when working in the games industry. It’s also useful to be confident switching between game engines, as companies often use their own engines what will be different from ones I have worked with before. Being able to switch between engines and learn them quickly would greatly improve my versatility in the industry. One core part of Unreal is its actor classes and the systems around them. Increasing my knowledge of this area would help me to more easily pick up Unreal when I have to use it.


\subsection{Action} 

Towards this end I will spend the first month of the Summer holidays researching these systems in Unreal’s documentation, as well as examining the sample projects in the engine to better understand how they function.


\section{UML for Efficient Planning}

Figuring out how to design the architecture of a project is something that often takes up a lot of my time as I am often reluctant to invest too much time into actual code before I am certain how the project should be designed. Because of this I sometimes lose time that could be spent coding, instead doing inefficient planning. UML diagrams can simplify the planning process and make it easy to refer back to, as well as being a useful documentation tool. They also allow multiple programmers to easily understand how the project is being created so that designs do not conflict. I feel that I could use UML more often to effectively plan my projects, allowing me to get into coding faster.


\subsection{Action} 

For each new feature of a project, I will make a simple UML diagram showing the Class Hierarchy, which will give me a firm starting point from which I can begin to write code. This will streamline my planning and could also be useful in the group game project, by allowing my fellow programmer and I to create a design we can refer back to. 


\section{Balance}

Maintaining the balance between my crushing existential dread at the notion of ‘work’, and my economically suicidal, hyperconfident egomania is a delicate affair. Recently – possibly due to my general state of intermittent sleep-deprivation – I have been leaning in favour of the crippling existential dread. This is obviously not ideal for productivity and also causes a lack of engagement in projects, leading to less time spent with the team. It is also harmful to my longer-term plans, as my resolute faith in my hypothetical ability to circumvent (and/or circumnavigate) a work environment is mostly what allows me to remain motivated. While it is important not to overcompensate and sway too heavily to towards all-consuming egomania, it is generally the more preferable of the two, as it allows me to function at a much more productive level.

\subsection{Action}

In order to regain an equilibrium in my precarious mental state, I will spend 10 minutes each morning and evening meditating. This will be done outside if possible, but indoors is acceptable. This will allow me to focus on calming my mind and engaging in work at a higher quality than I would otherwise.

\section{Conclusion}

Overall, I am confident that with these goals in mind I will be able to move into my third year of university with an improvement to my work, especially the group project and my ability to divide my time with less bias between projects. The skills and challenges I have highlighted in this report are all things that I believe are relevant to any future career in the industry that I may end up in, and I am eager to see myself grow in these areas.


\bibliographystyle{ieeetran}
%\bibliography{references}

\end{document}